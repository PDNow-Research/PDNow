 % use the "wcp" class option for workshop and conference
 % proceedings
 %\documentclass[gray]{jmlr} % test grayscale version
 %\documentclass[tablecaption=bottom]{jmlr}% journal article
 \documentclass[pmlr,twocolumn,10pt]{jmlr} % W&CP article

 \let\SUP\textsuperscript

% \usepackage{geometry}
% \geometry{margins=0.1in,textwidth=7in}

 % The following packages will be automatically loaded:
 % amsmath, amssymb, natbib, graphicx, url, algorithm2e

 %\usepackage{rotating}% for sideways figures and tables
 %\usepackage{longtable}% for long tables

 % The booktabs package is used by this sample document
 % (it provides \toprule, \midrule and \bottomrule).
 % Remove the next line if you don't require it.

\usepackage{booktabs}
 % The siunitx package is used by this sample document
 % to align numbers in a column by their decimal point.
 % Remove the next line if you don't require it.
\usepackage[load-configurations=version-1]{siunitx} % newer version 
%\usepackage{siunitx}

 % The following command is just for this sample document:
\newcommand{\cs}[1]{\texttt{\char`\\#1}}% remove this in your real article

% The following is to recognise equal contribution for authorship
\newcommand{\equal}[1]{{\hypersetup{linkcolor=black}\thanks{#1}}}

 % Define an unnumbered theorem just for this sample document for
 % illustrative purposes:
\theorembodyfont{\upshape}
\theoremheaderfont{\scshape}
\theorempostheader{:}
\theoremsep{\newline}
\newtheorem*{note}{Note}

 % change the arguments, as appropriate, in the following:
\jmlrvolume{LEAVE UNSET}
\jmlryear{2021}
\jmlrsubmitted{LEAVE UNSET}
\jmlrpublished{LEAVE UNSET}
\jmlrworkshop{Machine Learning for Health (ML4H) 2021} % W&CP title

 % The optional argument of \title is used in the header
\title[Short Title]{Full Title of Article\titlebreak This Title Has
A Line Break}

 % Authors with different addresses and equal first authors:
\author{Riya Tyagi \nametag{\thanks{Authors contributed equally}\SUP{1}},
Tanish Tyagi \nametag{\footnotemark[1]\SUP{2}}, 
Ming Wang \SUP{3},
Lijun Zhang \SUP{4},
\centering \Email{
\\[\bigskipamount] 
\SUP{1}\{rtyagi\}
@exeter.edu}
\centering \Email{
\\[\bigskipamount] 
\SUP{2}\{ttyagi\}@mgh.harvard.edu}
\centering \Email{
\\[\bigskipamount] 
\SUP{3}\{mwang\}@phs.psu.edu}
\centering \Email{
\\[\bigskipamount] 
\SUP{4}\{lzhang6\}@pennstatehealth.psu.edu}
}

\begin{document}

\maketitle

\begin{abstract}
Parkinson’s disease (PD) is debilitating, progressive, and clinically marked by motor symptoms, which affects over 10 million lives around the world. As the second most common neurodegenerative disease with no approved treatments, the existing diagnosis methods have limitations, such as the expense of visiting doctors and the challenge of automated early detection, considering that symptoms and behavioral differences in patients and healthy individuals are often not present in early stages. Micrographia is a disorder commonly observed in early stages of PD. In order to establish mechanisms for early detection of PD, we applied machine and deep learning techniques to extract signs of micrographia from drawing samples gathered from a variety of open-source datasets. We have discovered that a Convolutional Neural Network (CNN) performs best and can reliably learn important information solely from raw images. This work sets the foundations for a web portal allowing any individual worldwide with access to a pen, printer, and phone to upload an image of a handwriting assessment we designed and receive a PD diagnosis in real-time.
\end{abstract}
\begin{keywords}
CNN, Parkinson’s, PD, micrographia, disorder, early detection, diagnosis, neurodegenerative
\end{keywords}

\section{Introduction}
\label{sec:intro}

Parkinson’s disease (PD) is a degenerative, chronic, and progressive nervous system disorder that affects movement. More than 10 million people worldwide are currently living with PD, and the number of cases is expected to more than double in the next decade. Symptoms include tremors, slowness of movement, stiffness, and changes in writing skills. PD occurs when dopamine-producing nerve cells die, a process that can take almost 10 years to reach its late and most severe stages. A lack of dopamine results in further critical symptoms, such as depression, anxiety, sleep disturbances, and dementia. 

Parkinson’s disease cannot be treated, although specific drugs and medications can assist with symptoms in early stages, potentially preventing the disease’s progression to more severe stages. Early detection and prognosis of PD are crucial for assisting patients to retain a good quality of life. However, diagnosing Parkinson’s disease in its early stages is a very challenging task, and there are currently no specific tests designed to diagnose PD. Diagnosis is based on a thorough examination performed by a trained medical official, including a neurological examination, medical history evaluation, blood and laboratory tests, and brain scans. Even with these procedures, PD is misdiagnosed up to 30\% of the time, especially in early stages, due to the many Parkinson’s mimics, the main ones being Essential Tremors and Drug Induced Parkinson’s. Additionally, up to 20\% of PD patients are undiagnosed. Receiving a PD diagnosis is expensive, time consuming, and non-accessible, as reported by 21\% of patients who had to visit their general provider thrice before receiving a specialist referral for their condition. 

Tools that can make the diagnosis process accurate, accessible, automatic, real-time, early, and free of cost can have tremendous impact for Parkinson’s patients. In this study, we apply machine and deep learning strategies on a variety of data, including both image and numerical features. We present a Convolutional Neural Network model able to uncover signs of Parkinson’s in drawing images and compare it to machine learning models.

\section{Related Works}
\label{related-works}
Prior works have incorporated the use of voice recordings, electroencephalogram (EEG) signals, and smart devices to create models that deduce whether a patient has PD or not. Zhao et al. hypothesized that patients with PD exhibit deficits in the production of emotional speech. To test this conjecture, five patients and seven healthy individuals were used to recognize Parkinson’s disease through voice recordings. Naive Bayes, Random Forests, and Support Vector Machines were applied for classification, achieving 65.5\% and 73.33\% accuracies on classifying PD and control. Oh et al. utilized EEG signals and a Convolutional Neural Network (CNN) to detect PD
by assessing whether the EEG signals depicted brain abnormalities, achieving 88.25\% accuracy. Drotár et al. employed the use of a digitizing tablet to assess both in-air and on-surface kinematic variables during handwriting of a sentence in 37 PD patients on medication and 38 age- and gender-matched healthy controls. Using Support Vector Machines, an accuracy of 85.61\% was achieved. These studies require patients to visit a neurologist to access a smart device and undergo various scans like EEG and PET, making the process expensive, time-consuming, and difficult to access. In this study, we aim to make the PD diagnosis process accurate, efficient, and accessible by using images of handwriting samples to automate the PD diagnosis process. 

\section{Dataset and Preprocessing}
\label{sec:Dataset+Preprocessing}

\paragraph{Dataset}
\label{sec:Dataset} Our dataset was formed by combining the New HandPD and Old HandPD dataset available at (https://wwwp.fc.unesp.br/~papa/pub/datasets/Handpd/). It consists of 158 individuals, 53 healthy and 105 PD patients. Each patient drew 8 images, 4 spirals and 4 meander. With these images, we constructed 2 datasets, one consisting of extracted numerical features and the other consisting of images.

\paragraph{Preprocessing}
\label{sec:Preprocessing} 
We mimicked the feature extraction process used (cite handpd ppl) to extract 9 numerical features from the images. First, the handwritten trace (HT) and exam trace (ET) were extracted from the image. Figure (X) shows an example image and the extracted HT and ET.  Using the HT and ET, the below features were computed:

\begin{enumerate}
\item The root mean square (RMS) of the differences between the HT and ET radii. The radius of the HT or ET can be defined as the length of the straight line that connects an arbitrary point to the center of the HT or ET and is shown in figure (X). $N$ is the number of sample points on the HT and ET.
\[RMS = \sqrt{\frac{1}{N} * \sum_\SUP{i=1}^N(r_\SUP{HT}^i - r_\SUP{ET}^i)^2}\]

\item The maximum difference between the HT and ET radii. 
\[\Delta max = max(|r_\SUP{HT}^i - r_\SUP{ET}^i|)\]

\item F3: The minimum difference between the HT and ET radii. 
\[\Delta min = min(|r_\SUP{HT}^i - r_\SUP{ET}^i|)\]

\item F4: The standard deviation of the differences between the HT and ET radii. 
\[s = \sqrt{\frac{1}{N-1}\sum_{i=1}^N(r_\SUP{HT}^i - r_\SUP{ET}^i)^2}\]

\item F5: The Mean Relative Tremor (mrt) of a given individual’s HT. This is defined as the mean difference between the radius of a given sample and its D left-nearest neighbors. To find D, D = {1, 3, 5, 7, 10, 15, 20} were tested, with D = 10 maximizing the PD detection rate. 
\[MRT = \frac{1}{N-D}\sum_\SUP{i=D}^N(|r_\SUP{HT}^\SUP{i-D+1} - r_\SUP{ET}^i|)\]

Figures 6-8 are computed ultizing the equation for relative tremor: $r_\SUP{ET}^i - r_\SUP{HT}^\SUP{i-D+1}|$
\item F6: Maximum ET
\item F7: Minimum ET
\item F8: Standard Deviation of ET values
\item F9: The number of times the difference between HT and ET radii changes from positive to negative.
\end{enumerate}

\end{document}
